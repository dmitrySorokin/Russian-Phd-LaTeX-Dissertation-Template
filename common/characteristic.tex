Современные методы глубокого обучения с подкреплением (RL) способны решать задачи оптимального управления и планирования без использования априорной информации о решаемой задаче. Обучение происходит путем проб и ошибок в котором агент взаимодействует со средой и учится оптимизировать свои действия так, чтобы они приводили к большей ожидаемой награде. Такая формулировка метода машинного обучения наиболее близко отражает то, как учится человек и может рассматриваться как один из подходов к созданию общего искусственного интеллекта \cite{reward_is_enough}. Создание алгоритмов способных к принятию решений в окружающем нас мире, способных к самообучению и не требующих при этом больших объемов размеченных человеком данных способно вызвать взрывной рост в коллаборативной роботике, беспилотном транспорте и в области виртуальных ассистентов. В представленной работе разработаны методы управления робототехническими устройствами и виртуальными агентами на основе машинного обучения с подкреплением. Впервые в рамках данной работы был разработан метод автоматической настройки оптического интерферометра на основе машинного обучения с подкреплением. Автоматизация настройки прецизионного оптического оборудования позволит существенно ускорить проведение оптических экспериментов. Предложенный метод использует изображения с камеры и способен обучаться под параметры конкретной установки. Разработанный метод по качеству и скорости настройки интерферометра существенно превосходит человека. Также был разработан алгоритм, комбинирующий в себе обучение с подкреплением, классические алгоритмы на графах и экспертные знания, для управления виртуальным агентом в среде NetHack \cite{nethack}. Данная среда является одной из наиболее сложных тестовых сред для алгоритмов обучения с подкреплением. Разработанный метод позволил эффективно применить обучение с подкреплением в данной задаче, и занял первое место по результатам соревнования проводимого Google DeepMind и Facebook AI Research в рамках конференции NeurIPS Competition track 2021. 


{\actuality} Впервые глубокое обучение с подкреплением было применено для управления виртуальным агентом в среде основанной на играх ``Atari 2600'' в 2013 году\cite{mnih2013atari}. Было показано, что RL способен управлять агентом на основе визуальной информации на уровне человека. С этих пор теме машинного обучения с подкреплением уделяется все больше внимания. Обучение с подкреплением доказало свою эффективность обойдя человека во многих задачах, таких как игра в шахматы~\cite{alphazero}, игра Го~\cite{alphago} и StarCraft II~\cite{alphastar}. Эти впечатляющие результаты стали возможны не только благодаря развитию методов RL, но во многом из-за развития вычислительной техники, так как для успешного обучения агента требуется большое количество эпизодов взаимодействия со средой. 

Одним из наиболее перспективных приложений для методов машинного обучения с подкреплением является робототехника. Так современные роботы уже заменяют человека на производстве, так как в условиях строго контролируемого окружения возможно задать управляющую программу, учитывающую все возможные ситуации. В условиях же повседневной жизни, большую роль играет возможность действовать в условиях не определенности. В таких случаях оптимальное поведение сложно запрограммировать, но оно может быть выучено путем взаимодействия со средой. 
Однако для применения методов обучения с подкреплением в повседневной жизни нужно решить рад проблем таких как уменьшение количества примеров необходимых для обучения; повышение эффективности переноса моделей, обученных в симуляции на реальный мир; разработка алгоритмов способных решать задачи в условиях отсутствующей или очень редкой функции награды. 

В данной работе сделан фокус на применении методов глубокого обучения с подкреплением к решению прикладных задач. Одной из наиболее трудозатратных фаз при проведении оптического эксперимента является юстировка оборудования. В экспериментальной установке используются сотни оптических элементов таких как линзы, зеркала, аттенюаторы и др. Каждый из элементов должен быть отъюстирован с микрометрической точностью. Настройка оборудования требует большого экспериментального опыта и занимает много часов даже у опытного специалиста. В данной работе впервые решена задача автоматизации настройки оптического интерферометра методом машинного обучения с подкреплением. Оптические интерферометры являются составной частью большинства оптических установок используемых в экспериментальной работе. Автоматизация процесса настройки экспериментальной установки может существенно ускорить проведение научных исследований и уменьшить количество ручного труда. Разработанный метод настройки интерферометра не использует априорных знаний о задаче и способен самостоятельно обучаться юстировке интерферометров различной конструкции, геометрии, с разными параметрами оптических элементов. Агент обучается настраивать оптический интерферометр сначала в симуляции используя большое количество синтетических данных, а затем без дообучения может быть использован на экспериментальной установке. Использование симуляции позволяет производить обучение на большом количестве взаимодействий со средой. Высокое качество работы при переносе агента на экспериментальную установку достигается благодаря использованию рандомизаций среды при обучении агента. В этом случае экспериментальная установка для агента выступает в качестве одной из рандомизаций. 

В рамках данной работы рассматривается задача обучения агента для среды, основанной на компьютерной игре NetHack. Данная игра представляет собой один из самых серьезных вызовов для методов машинного обучения с подкреплением \cite{nethack}. Средняя длинна эпизода в NetHack составляет 100'000 шагов, что в 50 раз больше чем в StarCraft II. Также NetHack является процедурно генерируемой средой, из-за чего агент редко может оказаться в одном состоянии больше одного раза. Большое пространство действий и различных состояний среды приводит к тому, что большинство методов, используемых в обучении с подкреплением для исследования среды, не работают в такой постановке. В данной работе разработан иерархический метод управления агентом. В нем стратегия строится из базовых навыков предназначенных для решения конкретных задач, а выбор навыка происходит на основании текущего состояния. Навыки реализованы как с использованием обучения с подкреплением, так и с помощью классических алгоритмов и экспертных знаний. Данный подход похож на предложенный в статье \cite{Sutton1999} метод опционов. Разработанный гибридный метод построения агента позволил превзойти другие алгоритмы основанные на обучении с подкреплением. Данный подход может быть использован при проектировании систем сочетающих в себе машинное обучение и классические алгоритмы. 

Также в данной работе рассматривается задача управления движением шагающего робота с заданной скоростью. В случае робота с большим количеством степеней свободы управление с использованием классических алгоритмов представляет собой достаточно сложную задачу. В этом случае использование обучения с подкреплением имеет существенные преимущества так как агент может самостоятельно выучить оптимальную стратегию опираясь только на скалярную функцию награды. Оптимальная стратегия должна хорошо фильтровать шумы в наблюдениях, не требовать чрезмерно большого количества данных для обучения и не совершать действий, которые могли бы повредить роботу. В рамках данной работы был разработан подход для управления движением робота Unitree A1~\cite{unitree} основанный на обучении с подкреплением. При обучении стратегии используется расписание постоянно увеличивающее сложность текущей
задачи. Для обучения стратегии была разработана функция награды которая побуждает агента выучивать безопасное и плавное движение с заданной
скоростью. Результаты тестирования обученного агента показали, что он способен хорошо выполнять поставленные задачи в симуляции.

%\ifsynopsis
%% can add synopsis only text here
%\else
%Этот абзац появляется только в~диссертации.
%Через проверку условия \verb!\!\verb!ifsynopsis!, задаваемого в~основном %файле
%документа (\verb!dissertation.tex! для диссертации), можно сделать новую
%команду, обеспечивающую появление цитаты в~диссертации, %но~не~в~автореферате.
%\fi

% {\progress}
% Этот раздел должен быть отдельным структурным элементом по
% ГОСТ, но он, как правило, включается в описание актуальности
% темы. Нужен он отдельным структурынм элемементом или нет ---
% смотрите другие диссертации вашего совета, скорее всего не нужен.

{\aim} данной работы является развитие методов машинного обучения с подкреплением и применение их к задачам управления робототехническими устройствами и виртуальными агентами. 

Для~достижения поставленной цели необходимо было решить следующие {\tasks}:
\begin{enumerate}[beginpenalty=10000] % https://tex.stackexchange.com/a/476052/104425
  \item Разработка компьютерной модели оптического интерферометра Маха-Цендера
  \item Исследование применимости методов машинного обучения с подкреплением в задаче автоматизированной настройки оптического интерферометра
  \item Подбор функции награды; определение пространства состояний и действий; подбор гиперпараметров
  \item Обучение алгоритма машинного обучения настраивать интерферометр в симуляции
  \item Разработка программно-аппаратного комплекса для использования алгоритма при настройке физического интерферометра
  \item Разработка алгоритма для игры в NetHack
  \item Разработка метода управления движением шагающего робота с заданной линейной и угловой скоростями
\end{enumerate}


{\novelty}
\begin{enumerate}[beginpenalty=10000] % https://tex.stackexchange.com/a/476052/104425
  \item Впервые были разработаны методы настройки оптического интерферометра на основе машинного обучения с подкреплением с использованием дискретного и непрерывного пространства действий
  \item Впервые создан программно-аппаратный комплекс настройки оптического интерферометра по изображениям с камеры основанный на машинном обучении с подкреплением
  \item Было выполнено оригинальное исследование применимости иерархического алгоритма сочетающего в себе машинное обучения с подкреплением и запрограммированное поведение для игры Nethack
  \item Был разработан оригинальный метод обучения стратегии для управления движением шагающего робота с заданной линейной и угловой скоростью. 
\end{enumerate}

{\influence} работы заключается в следующем:
\begin{enumerate}[beginpenalty=10000] % https://tex.stackexchange.com/a/476052/104425
  \item Применение предложенного в работе автоматизированного подхода к настройке оптического интерферометра позволит существенно ускорить проведение физических экспериментов и снизит необходимость в ручном труде. 
  \item Разработанные алгоритмы для управления виртуальными агентами затем могут быть применены в робототехнике, самоуправляемых автомобилях и виртуальных ассистентах. 
\end{enumerate}

{\methods} При проведении работы использовались методы машинного обучения, компьютерного зрения, машинного обучения с подкреплением, разработки программного обеспечения, линейной алгебры, общей физики и оптики. 

{\defpositions}
\begin{enumerate}[beginpenalty=10000] % https://tex.stackexchange.com/a/476052/104425
  \item Был разработан метод настройки оптического интерферометра основанный на применении методов машинного обучения с подкреплением. Разработанный метод позволяет настраивать оптический интерферометр без участия человека основываясь исключительно на изображениях интерференционной картины. Предложенный метод не использует априорной информации и способен самостоятельно обучаться под конкретную установку.
  \item Был разработан программно-аппаратный комплекс автоматической настройки оптического интерферометра. Скорость и точность настройки с использованием разработанного метода существенно превосходят ручную настройку. 
  \item Был разработан иерархический алгоритм комбинирующий алгоритмический и нейросетевой подходы для управления агентом в среде NetHack.
  \item Был разработан метод обучения стратегии для управления движением шагающего робота с заданной линейной и угловой скоростью. 
\end{enumerate}

{\reliability} полученных результатов обеспечивается комплексным тестированием предложенного метода автоматизированной настройки оптического интерферометра, проведенной в ООО МЦКТ. По результатам 
соревнования проводимого Google DeepMind и Facebook AI Research разработанный метод управления агентом в среде NetHack превзашел остальные подходы использующие нейронные сети. 


{\probation} Основные результаты работы докладывались~на: 34-й международной конференции Neural Information Processing Systems (NeurIPS) в 2020 году (доклад был отмечен как spotlight); на 29-й ежегодной международной конференции по лазерной физике LPHYS'21; на 5-й международной конференции Conference on Robot Learning (CoRL) в 2021 году; на 35-й международной конференции Neural Information Processing Systems (NeurIPS, Competition track) в 2021 году. Международной конференции по квантовым технологиям ICQT в 2021 году; международной научно-технической конференции Нейроинформатика в 2022 году.

{\contribution} Автором лично разработан симулятор оптического интерферометра; программно-аппаратный комплекс для запуска и тестирования обученного агента на экспериментальной установке; метод настройки интерферометра с помощью алгоритма машинного обучения с подкреплением, использующий дискретное пространство действий. Автор принимал активное участие в разработке метода машинного обучения для настройки интерферометра использующего непрерывное пространство действий. Автором лично предложена и реализована идея алгоритма для игры в Nethack в виде иерархического агента, сочетающего в себе машинное обучение с подкреплением и алгоритмический подход. Автором лично предложена и реализована функция награды позволяющая обучить стратегию для управления движением шагающего робота с заданной скоростью. 

\ifnumequal{\value{bibliosel}}{0}
{%%% Встроенная реализация с загрузкой файла через движок bibtex8. (При желании, внутри можно использовать обычные ссылки, наподобие `\cite{vakbib1,vakbib2}`).
    {\publications} Основные результаты по теме диссертации изложены
    в~XX~печатных изданиях,
    X из которых изданы в журналах, рекомендованных ВАК,
    X "--- в тезисах докладов.
}%
{%%% Реализация пакетом biblatex через движок biber
    \begin{refsection}[bl-author, bl-registered]
        % Это refsection=1.
        % Процитированные здесь работы:
        %  * подсчитываются, для автоматического составления фразы "Основные результаты ..."
        %  * попадают в авторскую библиографию, при usefootcite==0 и стиле `\insertbiblioauthor` или `\insertbiblioauthorgrouped`
        %  * нумеруются там в зависимости от порядка команд `\printbibliography` в этом разделе.
        %  * при использовании `\insertbiblioauthorgrouped`, порядок команд `\printbibliography` в нём должен быть тем же (см. biblio/biblatex.tex)
        %
        % Невидимый библиографический список для подсчёта количества публикаций:
        \printbibliography[heading=nobibheading, section=1, env=countauthorvak,          keyword=biblioauthorvak]%
        \printbibliography[heading=nobibheading, section=1, env=countauthorwos,          keyword=biblioauthorwos]%
        \printbibliography[heading=nobibheading, section=1, env=countauthorscopus,       keyword=biblioauthorscopus]%
        \printbibliography[heading=nobibheading, section=1, env=countauthorconf,         keyword=biblioauthorconf]%
        \printbibliography[heading=nobibheading, section=1, env=countauthorother,        keyword=biblioauthorother]%
        \printbibliography[heading=nobibheading, section=1, env=countregistered,         keyword=biblioregistered]%
        \printbibliography[heading=nobibheading, section=1, env=countauthorpatent,       keyword=biblioauthorpatent]%
        \printbibliography[heading=nobibheading, section=1, env=countauthorprogram,      keyword=biblioauthorprogram]%
        \printbibliography[heading=nobibheading, section=1, env=countauthor,             keyword=biblioauthor]%
        \printbibliography[heading=nobibheading, section=1, env=countauthorvakscopuswos, filter=vakscopuswos]%
        \printbibliography[heading=nobibheading, section=1, env=countauthorscopuswos,    filter=scopuswos]%
        %
        \nocite{*}%
        %
        {\publications} Основные результаты по теме диссертации изложены в~\arabic{citeauthor}~печатных изданиях\sloppy
        \ifnum \value{citeauthorvak}>0% 
            \arabic{citeauthorvak} из которых изданы в журналах, рекомендованных ВАК\sloppy%
        \fi%
        \ifnum \value{citeauthorscopuswos}>0%
            , \arabic{citeauthorscopuswos} "--- в~периодических научных журналах, индексируемых Web of~Science и Scopus\sloppy%
        \fi%
        \ifnum \value{citeauthorconf}>0%
            , \arabic{citeauthorconf} "--- в~тезисах докладов.
        \else%
            .
        \fi%
        \ifnum \value{citeregistered}=1%
            \ifnum \value{citeauthorpatent}=1%
                Зарегистрирован \arabic{citeauthorpatent} патент.
            \fi%
            \ifnum \value{citeauthorprogram}=1%
                Зарегистрирована \arabic{citeauthorprogram} программа для ЭВМ.
            \fi%
        \fi%
        \ifnum \value{citeregistered}>1%
            Зарегистрированы\ %
            \ifnum \value{citeauthorpatent}>0%
            \formbytotal{citeauthorpatent}{патент}{}{а}{}\sloppy%
            \ifnum \value{citeauthorprogram}=0 . \else \ и~\fi%
            \fi%
            \ifnum \value{citeauthorprogram}>0%
            \formbytotal{citeauthorprogram}{программ}{а}{ы}{} для ЭВМ.
            \fi%
        \fi%
        % К публикациям, в которых излагаются основные научные результаты диссертации на соискание учёной
        % степени, в рецензируемых изданиях приравниваются патенты на изобретения, патенты (свидетельства) на
        % полезную модель, патенты на промышленный образец, патенты на селекционные достижения, свидетельства
        % на программу для электронных вычислительных машин, базу данных, топологию интегральных микросхем,
        % зарегистрированные в установленном порядке.(в ред. Постановления Правительства РФ от 21.04.2016 N 335)
    \end{refsection}%
    \begin{refsection}[bl-author, bl-registered]
        % Это refsection=2.
        % Процитированные здесь работы:
        %  * попадают в авторскую библиографию, при usefootcite==0 и стиле `\insertbiblioauthorimportant`.
        %  * ни на что не влияют в противном случае
        \nocite{progbib1}%program
        \nocite{confbib1}%conf
    \end{refsection}%
        %
        % Всё, что вне этих двух refsection, это refsection=0,
        %  * для диссертации - это нормальные ссылки, попадающие в обычную библиографию
        %  * для автореферата:
        %     * при usefootcite==0, ссылка корректно сработает только для источника из `external.bib`. Для своих работ --- напечатает "[0]" (и даже Warning не вылезет).
        %     * при usefootcite==1, ссылка сработает нормально. В авторской библиографии будут только процитированные в refsection=0 работы.
}
