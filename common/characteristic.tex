
{\actuality} В настоящее время методы машинного обучения широко используются в различных сферах жизни. Так машинное обучение помогает рекоммендовать товары~\cite{recsys}, переводить и классифицировать тексты на естественном языке~\cite{attention_is_all_you_need, bert}, и даже генерировать картины~\cite{dall-e}. Все это стало возможным благодяря большим объемам данных, росту вычислительных ресурсов и созданию новых алгоритмов машинного обучения. Следующим рубежом для методов машинного обучения вяляется область связанная с принятием решений. Создание алгоритмов способных к принятию решений в окружающем нас мире, способных к самообучению и не требующих при этом больших объемов размеченных человеком данных способно вызвать взрывной рост в коллаборативной роботике, беспилотном транспорте и в области виртуальных ассистентов. 

Одним из наиболее многообещающих методов для создания таких роботов является машинное обучение с подкреплением. Эти алгоритмы уже показали свою эффективность обойдя человека в некоторых задачах таких как игра в шахматы~\cite{alphazero} или Го~\cite{alphago}. Однако для применения таких алгоритмов в повседневной жизни нужно решить рад проблем таких как уменьшение количества примеров необходимых для обучения; повышение эффективности переноса моделей обученных в симуляции на реальный мир; разработка алгоритмов способных решать задачи в условиях отсутствующей или очень редкой функции награды. 

Частичному решению этих задач посвящена данная диссертация. В ней решается задача автоматизации настройки оптического интерферометра с использованием машинного обучения с подкреплением. Оптические интерферометры являются составной частью большинства оптических установок используемых в экспериментальной работе. Автоматизация процесса настройки экспериментальной установки может существенно ускорить проведение научных исследований и уменьшить количество ручного труда. Алгоритм машинного обучения обучается настраивать оптический интерферометр сначала в симуляции используя большое количество синтетических данных и рандомизаций среды, затем без дообучения переносится на реальный интерферометр. Также в данной работе рассматривается задача обучения агента для среды основанной на компьютерной игре Nethack. Данная игра представляет собой один из самых серьезных вызовов для методов машинного обучения с подкреплением \cite{nethack}. 

\ifsynopsis
% can add synopsis only text here
\else
Этот абзац появляется только в~диссертации.
Через проверку условия \verb!\!\verb!ifsynopsis!, задаваемого в~основном файле
документа (\verb!dissertation.tex! для диссертации), можно сделать новую
команду, обеспечивающую появление цитаты в~диссертации, но~не~в~автореферате.
\fi

% {\progress}
% Этот раздел должен быть отдельным структурным элементом по
% ГОСТ, но он, как правило, включается в описание актуальности
% темы. Нужен он отдельным структурынм элемементом или нет ---
% смотрите другие диссертации вашего совета, скорее всего не нужен.

{\aim} данной работы является развитие методов машинного обуения с подкреплением и применение их к задачам управления робототехническими устройствами и виртуальными агентами. 

Для~достижения поставленной цели необходимо было решить следующие {\tasks}:
\begin{enumerate}[beginpenalty=10000] % https://tex.stackexchange.com/a/476052/104425
  \item Разработать компьютерную модель оптического интерферометра Маха-Цендера
  \item Исследовать методы машинного обучения с подкреплением и их применимость к задаче автоматизированной настройки интерферометра
  \item Обучить алгоритм машинного обучения настраивать интерферометр в симуляцции. 
  \item Разработать программно-аппаратный комплекс для использования алгоритма при настройке физического интерферометра
  \item Протестировать качество работы алгоритма на физической установке и сравнить его с работой человека
  \item Разработать алгоритм для игры в Nethack
\end{enumerate}


{\novelty}
\begin{enumerate}[beginpenalty=10000] % https://tex.stackexchange.com/a/476052/104425
  \item Впервые были разработаны методы настройки оптического интерферометра с применением машинного обучения с подкреплением с использованием дискретного и непрерывныного пространства действий
  \item Впервые создан программно-аппаратный комплекс настройки оптрического интерферометра по изображениям с камеры основанный на машинном обучении с подкреплением
  \item Было выполнено оригинальное исследование применимости иерархического алгоритма сочетающего в себе машинное обучения с подкреплением и запрограммированное поведение для игры Nethack
\end{enumerate}

{\influence} работы заключается в следующем:
\begin{enumerate}[beginpenalty=10000] % https://tex.stackexchange.com/a/476052/104425
  \item Использование предложенного в работе автоматизированного подхода к настройке оптического интерферометра позволит существенно ускорить проведение физических экспериментов и снизит необходимость у ручном труде сотрудников. 
  \item Разработанные алгоритмы для управления виртуальными агентами затем могут быть применены в робототехнике, самооуправляемых автомобилях и виртуальных ассистентах. 
\end{enumerate}

{\methods} При проведении работы использовались методы машинного обучения, компьютерного зрения, машинного обучения с подкреплением, разработки программного обеспечения, линейной алгебры, общей физики и оптики. 

{\defpositions}
\begin{enumerate}[beginpenalty=10000] % https://tex.stackexchange.com/a/476052/104425
  \item Разработка метода настройки оптического интерферометра с использованеием методов машинного обучения с подкреплением
  \item Реализация метода автоматической растройки оптического интерферометра в виде программно-аппаратного комплекса
  \item Разработка иерархического агента для игры Nethack
\end{enumerate}

{\reliability} полученных результатов обеспечивается комплексным тестированием предложеннного метода автоматизированной настройки оптического интерферометра проведенной в ООО МЦКТ. 
Результаты находятся в соответствии с результатами, полученными другими авторами.


{\probation}
Основные результаты работы докладывались~на: 34-й международной конференции Neural Information Processing Systems (NeurIPS) в 2020 году; на 29-й ежегодной международной конференции по лазерной физике LPHYS'21; на 5-й международной конференции Conference on Robot Learning (CoRL) в 2021 году; на 35-й международной конференции Neural Information Processing Systems (NeurIPS) в 2021 году.

{\contribution} Автором лично разработан симулятор оптического интерферометра; программно аппаратный комплекс для запуска и тестирования обученного агента на экспериментальной установке; метод настройки интерферометра с помощъю алгоритма машинного обучения с подкреплением использующий дискретное пространство действий. Автор принимал активное участие в разработке метода машинного обучения для настройки интерферометра использующего непрерывное пространство дейсвий. Автором лично предложена идея реализации алгоритма для игры в Nethack в виде иерархического агента сочетающего в себе машинное обучение с подкреплением и алгоритмический подход.  

\ifnumequal{\value{bibliosel}}{0}
{%%% Встроенная реализация с загрузкой файла через движок bibtex8. (При желании, внутри можно использовать обычные ссылки, наподобие `\cite{vakbib1,vakbib2}`).
    {\publications} Основные результаты по теме диссертации изложены
    в~XX~печатных изданиях,
    X из которых изданы в журналах, рекомендованных ВАК,
    X "--- в тезисах докладов.
}%
{%%% Реализация пакетом biblatex через движок biber
    \begin{refsection}[bl-author, bl-registered]
        % Это refsection=1.
        % Процитированные здесь работы:
        %  * подсчитываются, для автоматического составления фразы "Основные результаты ..."
        %  * попадают в авторскую библиографию, при usefootcite==0 и стиле `\insertbiblioauthor` или `\insertbiblioauthorgrouped`
        %  * нумеруются там в зависимости от порядка команд `\printbibliography` в этом разделе.
        %  * при использовании `\insertbiblioauthorgrouped`, порядок команд `\printbibliography` в нём должен быть тем же (см. biblio/biblatex.tex)
        %
        % Невидимый библиографический список для подсчёта количества публикаций:
        \printbibliography[heading=nobibheading, section=1, env=countauthorvak,          keyword=biblioauthorvak]%
        \printbibliography[heading=nobibheading, section=1, env=countauthorwos,          keyword=biblioauthorwos]%
        \printbibliography[heading=nobibheading, section=1, env=countauthorscopus,       keyword=biblioauthorscopus]%
        \printbibliography[heading=nobibheading, section=1, env=countauthorconf,         keyword=biblioauthorconf]%
        \printbibliography[heading=nobibheading, section=1, env=countauthorother,        keyword=biblioauthorother]%
        \printbibliography[heading=nobibheading, section=1, env=countregistered,         keyword=biblioregistered]%
        \printbibliography[heading=nobibheading, section=1, env=countauthorpatent,       keyword=biblioauthorpatent]%
        \printbibliography[heading=nobibheading, section=1, env=countauthorprogram,      keyword=biblioauthorprogram]%
        \printbibliography[heading=nobibheading, section=1, env=countauthor,             keyword=biblioauthor]%
        \printbibliography[heading=nobibheading, section=1, env=countauthorvakscopuswos, filter=vakscopuswos]%
        \printbibliography[heading=nobibheading, section=1, env=countauthorscopuswos,    filter=scopuswos]%
        %
        \nocite{*}%
        %
        {\publications} Основные результаты по теме диссертации изложены в~\arabic{citeauthor}~печатных изданиях,
        \arabic{citeauthorvak} из которых изданы в журналах, рекомендованных ВАК\sloppy%
        \ifnum \value{citeauthorscopuswos}>0%
            , \arabic{citeauthorscopuswos} "--- в~периодических научных журналах, индексируемых Web of~Science и Scopus\sloppy%
        \fi%
        \ifnum \value{citeauthorconf}>0%
            , \arabic{citeauthorconf} "--- в~тезисах докладов.
        \else%
            .
        \fi%
        \ifnum \value{citeregistered}=1%
            \ifnum \value{citeauthorpatent}=1%
                Зарегистрирован \arabic{citeauthorpatent} патент.
            \fi%
            \ifnum \value{citeauthorprogram}=1%
                Зарегистрирована \arabic{citeauthorprogram} программа для ЭВМ.
            \fi%
        \fi%
        \ifnum \value{citeregistered}>1%
            Зарегистрированы\ %
            \ifnum \value{citeauthorpatent}>0%
            \formbytotal{citeauthorpatent}{патент}{}{а}{}\sloppy%
            \ifnum \value{citeauthorprogram}=0 . \else \ и~\fi%
            \fi%
            \ifnum \value{citeauthorprogram}>0%
            \formbytotal{citeauthorprogram}{программ}{а}{ы}{} для ЭВМ.
            \fi%
        \fi%
        % К публикациям, в которых излагаются основные научные результаты диссертации на соискание учёной
        % степени, в рецензируемых изданиях приравниваются патенты на изобретения, патенты (свидетельства) на
        % полезную модель, патенты на промышленный образец, патенты на селекционные достижения, свидетельства
        % на программу для электронных вычислительных машин, базу данных, топологию интегральных микросхем,
        % зарегистрированные в установленном порядке.(в ред. Постановления Правительства РФ от 21.04.2016 N 335)
    \end{refsection}%
    \begin{refsection}[bl-author, bl-registered]
        % Это refsection=2.
        % Процитированные здесь работы:
        %  * попадают в авторскую библиографию, при usefootcite==0 и стиле `\insertbiblioauthorimportant`.
        %  * ни на что не влияют в противном случае
        \nocite{progbib1}%program
        \nocite{confbib1}%conf
    \end{refsection}%
        %
        % Всё, что вне этих двух refsection, это refsection=0,
        %  * для диссертации - это нормальные ссылки, попадающие в обычную библиографию
        %  * для автореферата:
        %     * при usefootcite==0, ссылка корректно сработает только для источника из `external.bib`. Для своих работ --- напечатает "[0]" (и даже Warning не вылезет).
        %     * при usefootcite==1, ссылка сработает нормально. В авторской библиографии будут только процитированные в refsection=0 работы.
}
