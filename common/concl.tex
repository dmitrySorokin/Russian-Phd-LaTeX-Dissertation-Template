%% Согласно ГОСТ Р 7.0.11-2011:
%% 5.3.3 В заключении диссертации излагают итоги выполненного исследования, рекомендации, перспективы дальнейшей разработки темы.
%% 9.2.3 В заключении автореферата диссертации излагают итоги данного исследования, рекомендации и перспективы дальнейшей разработки темы.
\begin{enumerate}
  \item Была разработана компьютерная модель оптического интерферометра Маха-Цендера. На основании компьютерной модели была разработана среда для обучения агентов машинного обучения с подкреплением по настройке оптического интерферометра. 
  \item Процесс настройки интерферометра был представлен в виде Марковского процесса принятия решений, была определена функция награды, пространства состояний и действий агентов. 
  \item Алгоритмы машинного обучения с подкреплением были реализованы и обучены настройке интерферометра в симуляции. Затем обученные агенты были успешно протестированы при настройке экспериментальной установки интерферометра. Качество настройки интерферометра оказалось существенно выше, чем у человека. 
  \item Был разработан метод включающий в себя обучения с подкреплением для управлением виртуальным агентом в среде Nethack. 
\end{enumerate}

\newpage
