\begin{frame}[noframenumbering,plain]
    \setcounter{framenumber}{1}
    \maketitle
\end{frame}

\begin{frame}
    \frametitle{Цели работы:}
    \begin{itemize}
        \item Развитие методов машинного обучения с подкреплением для решения задач управления робототехническими устройствами.
    \end{itemize}

    \vspace{20pt}
    В данный момент область применения методов RL ограничена:
    \begin{itemize}
        \item Sim2Real gap, при переносе агента, обученного в симуляции, на физическую установку.
        \item Сложности с использованием действий различной амплитуды.
        \item Сходимость стратегии к локальному  оптимуму при не оптимальной функции награды.
       \item Сложности с объединением RL и классических алгоритмов в рамках иерархического агента.
    \end{itemize}
\end{frame}

\begin{frame}
    \frametitle{Положения, выносимые на защиту}
    \begin{itemize}
        \item Метод обучения с подкреплением способный оперировать действиями различного масштаба, устойчивый к оптическим шумам, и его применение для настройки оптического интерферометра.
        \item Программно-аппаратный комплекс автоматической настройки оптического интерферометра.
        \item Метод обучения стратегии для управления движением шагающего робота с заданной линейной и угловой скоростью.
        \item Иерархический алгоритм комбинирующий алгоритмический и нейросетевой подходы и его применение для управления агентом в среде NetHack.
    \end{itemize}
\end{frame}
\note{
    Проговариваются вслух положения, выносимые на защиту
}

\begin{frame}
    \frametitle{Содержание}
    \tableofcontents
\end{frame}
\note{
    Работа состоит из четырёх глав.

    \medskip
    В первой главе \dots

    Во второй главе \dots

    Третья глава посвящена \dots

    В четвёртой главе \dots
}
