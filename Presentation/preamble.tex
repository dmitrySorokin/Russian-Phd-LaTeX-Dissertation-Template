\begin{frame}[noframenumbering,plain]
    \setcounter{framenumber}{1}
    \maketitle
\end{frame}

\begin{frame}
    \frametitle{Положения, выносимые на защиту}
    \begin{itemize}
        \item Метод настройки оптического интерферометра основанный на применении методов машинного обучения с подкреплением
        \item Программно-аппаратный комплекс автоматической настройки оптического интерферометра
        \item Иерархический алгоритм комбинирующий алгоритмический и нейросетевой подходы для управления агентом в среде NetHack
        \item Метод обучения стратегии для управления движением шагающего робота с заданной линейной и угловой скоростью
    \end{itemize}
\end{frame}
\note{
    Проговариваются вслух положения, выносимые на защиту
}
\begin{frame}
    \frametitle{Содержание}
    \tableofcontents
\end{frame}
\note{
    Работа состоит из четырёх глав.

    \medskip
    В первой главе \dots

    Во второй главе \dots

    Третья глава посвящена \dots

    В четвёртой главе \dots
}
