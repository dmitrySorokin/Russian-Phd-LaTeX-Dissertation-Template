%% Согласно ГОСТ Р 7.0.11-2011:
%% 5.3.3 В заключении диссертации излагают итоги выполненного исследования, рекомендации, перспективы дальнейшей разработки темы.
%% 9.2.3 В заключении автореферата диссертации излагают итоги данного исследования, рекомендации и перспективы дальнейшей разработки темы.
\begin{enumerate}
  \item Была разработана компьютерная модель оптического интерферометра Маха-Цендера. На основании компьютерной модели была разработана среда для обучения агентов машинного обучения с подкреплением настройке оптического интерферометра. Среда позволяет моделировать интерференционные картины, получаемые в интерферометре Маха-Цендера с произвольным расположением зеркал и линз. 
  \item Процесс настройки интерферометра был представлен в виде Марковского процесса принятия решений, была определена функция награды, пространства состояний и действий агентов. Предложенная функция награды позволяет различать состояния с высокой видностью, благодаря чему агент способен производить точную настройку интерферометра. Было предложено преобразование пространства действий, позволяющее агенту, оперирующему непрерывными действиями, одинаково хорошо выбирать действия с разными порядками амплитуды. 
  \item Были разработаны и реализованы алгоритмы машинного обучения с подкреплением, способные оперировать действиями различного масштаба и устойчивые к оптическим шумам. 
  \item Разработанные алгоритмы были обучены и успешно протестированы на экспериментальной установке. Качество настройки интерферометра у агента, оперирующего непрерывным пространством действий, оказалось существенно выше, чем у человека. 
  \item Был разработан метод обучения стратегии для управления движением шагающего робота с заданной линейной и угловой скоростью. Благодаря предложенной функции награды получилось обучить стратегию, способную успешно управлять движением робота Unitree A1 с заданной скоростью. 
  \item Был разработан гибридный нейро-символьный метод для управления виртуальным агентом в среде NetHack, включающий в себя обучение с подкреплением. Предложенный метод показал свою эффективность в рамках соревнования Neurips NetHack Challange и занял первое место среди алгоритмов, использующих нейронные сети.
\end{enumerate}

\newpage
