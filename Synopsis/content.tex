\pdfbookmark{Общая характеристика работы}{characteristic}             % Закладка pdf
\section*{Общая характеристика работы}

\newcommand{\actuality}{\pdfbookmark[1]{Актуальность}{actuality}\underline{\textbf{\actualityTXT}}}
\newcommand{\progress}{\pdfbookmark[1]{Разработанность темы}{progress}\underline{\textbf{\progressTXT}}}
\newcommand{\aim}{\pdfbookmark[1]{Цели}{aim}\underline{{\textbf\aimTXT}}}
\newcommand{\tasks}{\pdfbookmark[1]{Задачи}{tasks}\underline{\textbf{\tasksTXT}}}
\newcommand{\aimtasks}{\pdfbookmark[1]{Цели и задачи}{aimtasks}\aimtasksTXT}
\newcommand{\novelty}{\pdfbookmark[1]{Научная новизна}{novelty}\underline{\textbf{\noveltyTXT}}}
\newcommand{\influence}{\pdfbookmark[1]{Практическая значимость}{influence}\underline{\textbf{\influenceTXT}}}
\newcommand{\methods}{\pdfbookmark[1]{Методология и методы исследования}{methods}\underline{\textbf{\methodsTXT}}}
\newcommand{\defpositions}{\pdfbookmark[1]{Положения, выносимые на защиту}{defpositions}\underline{\textbf{\defpositionsTXT}}}
\newcommand{\reliability}{\pdfbookmark[1]{Достоверность}{reliability}\underline{\textbf{\reliabilityTXT}}}
\newcommand{\probation}{\pdfbookmark[1]{Апробация}{probation}\underline{\textbf{\probationTXT}}}
\newcommand{\contribution}{\pdfbookmark[1]{Личный вклад}{contribution}\underline{\textbf{\contributionTXT}}}
\newcommand{\publications}{\pdfbookmark[1]{Публикации}{publications}\underline{\textbf{\publicationsTXT}}}

Современные методы глубокого обучения с подкреплением (RL) способны решать задачи оптимального управления и планирования без использования априорной информации о решаемой задаче. Обучение происходит путем проб и ошибок в котором агент взаимодействует со средой и учится оптимизировать свои действия так, чтобы они приводили к большей ожидаемой награде. Такая формулировка метода машинного обучения наболее близко отражает то, как учится человек и может рассматриваться как один из подходов к созданию общего искусственного интеллекта \cite{reward_is_enough}. Создание алгоритмов способных к принятию решений в окружающем нас мире, способных к самообучению и не требующих при этом больших объемов размеченных человеком данных способно вызвать взрывной рост в коллаборативной роботике, беспилотном транспорте и в области виртуальных ассистентов. В представленной работе разработаны методы управления робототехническими устройствами и виртуальными агентами на основе машинного обучения с подкреплением. Впервые в рамках данной работы был разработан метод автоматической настройки оптического интерферометра на основе машинного обучения с подкреплением. Автоматизация настройки прецезинонного оптического оборудования позволит существенно ускорить проведение оптических экспериментов. Предложенный метод использует изображения с камеры и способен обучаться под параметры конкретной установки. Разработанный метод по качеству и скорости настройки интерферометра существенно превосходит человека. Также был разработан алгоритм, комбинирующий в себе обучение с подкреплением, классические алгоритмы на графах и экспертные знания, для управления виртуальным агентом в среде NetHack \cite{nethack}. Данная среда является одной из наиболее сложных тестовых сред для алгоритмов обучения с подкреплением. Разработанный метод позволил эффективно применить обучение с подкреплением в данной задаче, и занял первое место по результатам соревнования проводимого Google DeepMind и Facebook AI Research в рамках конференции NeurIPS Competition track 2021. 


{\actuality} Впервые глубокое обучение с подкреплением было применено для управления виртуальным агентом в среде основанной на играх ``Atari 2600'' в 2013 году\cite{mnih2013atari}. Было показано, что RL способен управлять агентом на основе визуальной информации на уровне человека. С этих пор теме машинного обучения с подкреплением уделяется все больше внимания. Обучение с подкреплением доказало свою эффективность обойдя человека во многих задачах, таких как игра в шахматы~\cite{alphazero}, игра Го~\cite{alphago} и StarCraft II~\cite{alphastar}. Эти впечатляющие результаты стали возможны не только благодяря развитию методов RL, но во многом из-за развития вычислительной техники, так как для успешного обучения агента требуется большое количество эпизодов взаимодействия со средой. 

Одним из наиболее перспективных приложений для методов машинного обучения с покреплением является робототехника. Так современные роботы уже заменяют человека на производстве, так как в условиях строго контролируемого окружения возможно задать управляющую программу учитывающую все возможные ситуации. В условиях же повседневной жизни, большую роль играет возможность действовать в условиях не определенности. В таких случаях оптимальное поведение сложно запрограммировать, но оно может быть выучено путем взаимодействия со средой. 
Однако для применения методов обучения с подкреплением в повседневной жизни нужно решить рад проблем таких как уменьшение количества примеров необходимых для обучения; повышение эффективности переноса моделей, обученных в симуляции на реальный мир; разработка алгоритмов способных решать задачи в условиях отсутствующей или очень редкой функции награды. 

В данной работе сделан фокус на применении методов глубокого обучения с подкреплением к решению прикладных задач. Одной из наиболее трудозатратных фаз при проведении оптического эксперимента является юстировка оборудования. В экспериментальной установке используются сотни оптических элементов таких как линзы, зеркала, аттенюаторы и др. Каждый из элементов должен быть отюстирован с микрометрической точностью. Настройка оборудования требует большого экспериментального опыта и занимает много часов даже у опытного специалиста. В данной работе впервые решена задача автоматизации настройки оптического интерферометра методом машинного обучения с подкреплением. Оптические интерферометры являются составной частью большинства оптических установок используемых в экспериментальной работе. Автоматизация процесса настройки экспериментальной установки может существенно ускорить проведение научных исследований и уменьшить количество ручного труда. Разработанный метод настройки интерферометра не использует априорных знаний о задаче и способен самостоятельно обучаться юстировке интерферометров различной конструкции, геометрии, с разными параметрами оптичнских элементов. Агент обучается настраивать оптический интерферометр сначала в симуляции используя большое количество синтетических данных, а затем без дообучения может быть использован на экспериментальной установке. Использование симуляции позволяет производить обучение на большом количестве взаимодействий со средой. Высокое качество работы при переносе агента на экспериментальную установку достигается благодаря использованию рандомизаций среды при обучении агента. В этом случае экспериментальная установка для агента выступает в качестве одной из рандомизаций. 

Также в рамках данной работы рассматривается задача обучения агента для среды, основанной на компьютерной игре NetHack. Данная игра представляет собой один из самых серьезных вызовов для методов машинного обучения с подкреплением \cite{nethack}. Средняя длинна эпизода в NetHack составляет 100'000 шагов, что в 50 раз больше чем в StarCraft II. Также NetHack является процедурно генерируемой средой, из-за чего агент редко может оказаться в одном состоянии больше одного раза. Большое пространство действий и различных состояний среды приводит к тому, что большинство методов, используемых в обучении с с подкреплением для исследования среды, не работают в такой постановке. В данной работе разработан иерархический метод управления агентом. В нем стратегия строится из базовых навыков предназначенных для решения конкретных задач, а выбор навыка происходит на основании текущего состояния. Навыки реализованы как с использованием обучения с подкреплением, так и с помощъю классических алгоритмов и экспертных знаний. Данный подход похож на предложенный в статье \cite{Sutton1999} метод опционов. Разработанный гибридный метод построения агента позволил превзойти другие алгоритмы основанные на обучении с подкреплением. Данный подход может быть использован при проектировании систем сочетающих в себе машинное обучение и класические алгоритмы. 

\ifsynopsis
% can add synopsis only text here
\else
Этот абзац появляется только в~диссертации.
Через проверку условия \verb!\!\verb!ifsynopsis!, задаваемого в~основном файле
документа (\verb!dissertation.tex! для диссертации), можно сделать новую
команду, обеспечивающую появление цитаты в~диссертации, но~не~в~автореферате.
\fi

% {\progress}
% Этот раздел должен быть отдельным структурным элементом по
% ГОСТ, но он, как правило, включается в описание актуальности
% темы. Нужен он отдельным структурынм элемементом или нет ---
% смотрите другие диссертации вашего совета, скорее всего не нужен.

{\aim} данной работы является развитие методов машинного обучения с подкреплением и применение их к задачам управления робототехническими устройствами и виртуальными агентами. 

Для~достижения поставленной цели необходимо было решить следующие {\tasks}:
\begin{enumerate}[beginpenalty=10000] % https://tex.stackexchange.com/a/476052/104425
  \item Разработка компьютерной модели оптического интерферометра Маха-Цендера
  \item Исследование применимости методов машинного обучения с подкреплением в задаче автоматизированной настройки оптического интерферометра
  \item Подбор функции награды; определение пространства состояний и действий; подбор гиперпараметров
  \item Обучение алгоритма машинного обучения настраивать интерферометр в симуляцяии
  \item Разработка программно-аппаратного комплекса для использования алгоритма при настройке физического интерферометра
  \item Разработка алгоритма для игры в NetHack
\end{enumerate}


{\novelty}
\begin{enumerate}[beginpenalty=10000] % https://tex.stackexchange.com/a/476052/104425
  \item Впервые были разработаны методы настройки оптического интерферометра на основе машинного обучения с подкреплением с использованием дискретного и непрерывныного пространства действий
  \item Впервые создан программно-аппаратный комплекс настройки оптического интерферометра по изображениям с камеры основанный на машинном обучении с подкреплением
  \item Было выполнено оригинальное исследование применимости иерархического алгоритма сочетающего в себе машинное обучения с подкреплением и запрограммированное поведение для игры Nethack
\end{enumerate}

{\influence} работы заключается в следующем:
\begin{enumerate}[beginpenalty=10000] % https://tex.stackexchange.com/a/476052/104425
  \item Применение предложенного в работе автоматизированного подхода к настройке оптического интерферометра позволит существенно ускорить проведение физических экспериментов и снизит необходимость в ручном труде. 
  \item Разработанные алгоритмы для управления виртуальными агентами затем могут быть применены в робототехнике, самоуправляемых автомобилях и виртуальных ассистентах. 
\end{enumerate}

{\methods} При проведении работы использовались методы машинного обучения, компьютерного зрения, машинного обучения с подкреплением, разработки программного обеспечения, линейной алгебры, общей физики и оптики. 

{\defpositions}
\begin{enumerate}[beginpenalty=10000] % https://tex.stackexchange.com/a/476052/104425
  \item Был разработан метод настройки оптического интерферометра основанный на применении методов машинного обучения с подкреплением. Разработанный метод позволяет настраивать оптический интерферометр без участия человека основываясь исключительно на изображениях интерференционной картины. Предложенный метод не использует априорной информации и способен самостоятельно обучаться под конкретную установку.
  \item Был разработан программно-аппаратный комплекс автоматической настройки оптического интерферометра. Скорость и точность настройки с использованием разработанного метода существенно превосходят ручную настройку. 
  \item Был разработан иерархический алгоритм комбинирующий алгоритмический и нейросетевой подходы для управления агентом в среде NetHack.
\end{enumerate}

{\reliability} полученных результатов обеспечивается комплексным тестированием предложенного метода автоматизированной настройки оптического интерферометра, проведенной в ООО МЦКТ. По результатам 
соревнования проводимого Google DeepMind и Facebook AI Research разработанный метод управления агентом в среде NetHack превзашел остальные подходы использующие нейронные сети. 


{\probation} Основные результаты работы докладывались~на: 34-й международной конференции Neural Information Processing Systems (NeurIPS) в 2020 году (доклад был отмечен как spotlight); на 29-й ежегодной международной конференции по лазерной физике LPHYS'21; на 5-й международной конференции Conference on Robot Learning (CoRL) в 2021 году; на 35-й международной конференции Neural Information Processing Systems (NeurIPS, Competition track) в 2021 году. Международной конференции по квантовым технологиям ICQT в 2021 году.

{\contribution} Автором лично разработан симулятор оптического интерферометра; программно-аппаратный комплекс для запуска и тестирования обученного агента на экспериментальной установке; метод настройки интерферометра с помощью алгоритма машинного обучения с подкреплением, использующий дискретное пространство действий. Автор принимал активное участие в разработке метода машинного обучения для настройки интерферометра использующего непрерывное пространство действий. Автором лично предложена и реализована идея алгоритма для игры в Nethack в виде иерархического агента, сочетающего в себе машинное обучение с подкреплением и алгоритмический подход.  

\ifnumequal{\value{bibliosel}}{0}
{%%% Встроенная реализация с загрузкой файла через движок bibtex8. (При желании, внутри можно использовать обычные ссылки, наподобие `\cite{vakbib1,vakbib2}`).
    {\publications} Основные результаты по теме диссертации изложены
    в~XX~печатных изданиях,
    X из которых изданы в журналах, рекомендованных ВАК,
    X "--- в тезисах докладов.
}%
{%%% Реализация пакетом biblatex через движок biber
    \begin{refsection}[bl-author, bl-registered]
        % Это refsection=1.
        % Процитированные здесь работы:
        %  * подсчитываются, для автоматического составления фразы "Основные результаты ..."
        %  * попадают в авторскую библиографию, при usefootcite==0 и стиле `\insertbiblioauthor` или `\insertbiblioauthorgrouped`
        %  * нумеруются там в зависимости от порядка команд `\printbibliography` в этом разделе.
        %  * при использовании `\insertbiblioauthorgrouped`, порядок команд `\printbibliography` в нём должен быть тем же (см. biblio/biblatex.tex)
        %
        % Невидимый библиографический список для подсчёта количества публикаций:
        \printbibliography[heading=nobibheading, section=1, env=countauthorvak,          keyword=biblioauthorvak]%
        \printbibliography[heading=nobibheading, section=1, env=countauthorwos,          keyword=biblioauthorwos]%
        \printbibliography[heading=nobibheading, section=1, env=countauthorscopus,       keyword=biblioauthorscopus]%
        \printbibliography[heading=nobibheading, section=1, env=countauthorconf,         keyword=biblioauthorconf]%
        \printbibliography[heading=nobibheading, section=1, env=countauthorother,        keyword=biblioauthorother]%
        \printbibliography[heading=nobibheading, section=1, env=countregistered,         keyword=biblioregistered]%
        \printbibliography[heading=nobibheading, section=1, env=countauthorpatent,       keyword=biblioauthorpatent]%
        \printbibliography[heading=nobibheading, section=1, env=countauthorprogram,      keyword=biblioauthorprogram]%
        \printbibliography[heading=nobibheading, section=1, env=countauthor,             keyword=biblioauthor]%
        \printbibliography[heading=nobibheading, section=1, env=countauthorvakscopuswos, filter=vakscopuswos]%
        \printbibliography[heading=nobibheading, section=1, env=countauthorscopuswos,    filter=scopuswos]%
        %
        \nocite{*}%
        %
        {\publications} Основные результаты по теме диссертации изложены в~\arabic{citeauthor}~печатных изданиях\sloppy
        \ifnum \value{citeauthorvak}>0% 
            \arabic{citeauthorvak} из которых изданы в журналах, рекомендованных ВАК\sloppy%
        \fi%
        \ifnum \value{citeauthorscopuswos}>0%
            , \arabic{citeauthorscopuswos} "--- в~периодических научных журналах, индексируемых Web of~Science и Scopus\sloppy%
        \fi%
        \ifnum \value{citeauthorconf}>0%
            , \arabic{citeauthorconf} "--- в~тезисах докладов.
        \else%
            .
        \fi%
        \ifnum \value{citeregistered}=1%
            \ifnum \value{citeauthorpatent}=1%
                Зарегистрирован \arabic{citeauthorpatent} патент.
            \fi%
            \ifnum \value{citeauthorprogram}=1%
                Зарегистрирована \arabic{citeauthorprogram} программа для ЭВМ.
            \fi%
        \fi%
        \ifnum \value{citeregistered}>1%
            Зарегистрированы\ %
            \ifnum \value{citeauthorpatent}>0%
            \formbytotal{citeauthorpatent}{патент}{}{а}{}\sloppy%
            \ifnum \value{citeauthorprogram}=0 . \else \ и~\fi%
            \fi%
            \ifnum \value{citeauthorprogram}>0%
            \formbytotal{citeauthorprogram}{программ}{а}{ы}{} для ЭВМ.
            \fi%
        \fi%
        % К публикациям, в которых излагаются основные научные результаты диссертации на соискание учёной
        % степени, в рецензируемых изданиях приравниваются патенты на изобретения, патенты (свидетельства) на
        % полезную модель, патенты на промышленный образец, патенты на селекционные достижения, свидетельства
        % на программу для электронных вычислительных машин, базу данных, топологию интегральных микросхем,
        % зарегистрированные в установленном порядке.(в ред. Постановления Правительства РФ от 21.04.2016 N 335)
    \end{refsection}%
    \begin{refsection}[bl-author, bl-registered]
        % Это refsection=2.
        % Процитированные здесь работы:
        %  * попадают в авторскую библиографию, при usefootcite==0 и стиле `\insertbiblioauthorimportant`.
        %  * ни на что не влияют в противном случае
        \nocite{progbib1}%program
        \nocite{confbib1}%conf
    \end{refsection}%
        %
        % Всё, что вне этих двух refsection, это refsection=0,
        %  * для диссертации - это нормальные ссылки, попадающие в обычную библиографию
        %  * для автореферата:
        %     * при usefootcite==0, ссылка корректно сработает только для источника из `external.bib`. Для своих работ --- напечатает "[0]" (и даже Warning не вылезет).
        %     * при usefootcite==1, ссылка сработает нормально. В авторской библиографии будут только процитированные в refsection=0 работы.
}
 % Характеристика работы по структуре во введении и в автореферате не отличается (ГОСТ Р 7.0.11, пункты 5.3.1 и 9.2.1), потому её загружаем из одного и того же внешнего файла, предварительно задав форму выделения некоторым параметрам

Диссертационная работа была выполнена при поддержке гранта УМНИК № 120ГУЦЭС8-D3/56352 от 21.12.2019. 

%\underline{\textbf{Объем и структура работы.}} Диссертация состоит из~введения,
%четырех глав, заключения и~приложения. Полный объем диссертации
%\textbf{ХХХ}~страниц текста с~\textbf{ХХ}~рисунками и~5~таблицами. Список
%литературы содержит \textbf{ХХX}~наименование.

\pdfbookmark{Содержание работы}{description}                          % Закладка pdf
\section*{Содержание работы}
Во \underline{\textbf{введении}} обосновывается актуальность
исследований, проводимых в~рамках данной диссертационной работы. Приводится обзор литературы показывающий большой потенциал методов машинного ообучения с подкреплением в задачах связанных с принятием решений. Формулируется цель и ставятся задачи данной работы. Излагается научная новизна
и практическая значимость научных результатов представляемой работы. 
В~последующих главах сначала описывается общий принцип, работы методов машинного обучения с подкреплением. Затем рассматриваются современные методы машинного обучения с подкреплением позволяющие достигать наилучших результатов в тестовых задачах. Потом приводится описание задачи настройки оптического интерферометра в виде задачи машинного обучения с подкреплением. Приводятся результаты обучения модели в симуляции и последующего тестирования разработанной модели на экспериментальной установке. В следующей главе формулируется необходимость в сложных тестовых задачах для развития и тестирования обобщающей способности методов машинного обучения с подкреплением. Формулируеются отличительные особенности позволяющие считать среду основанную на игре Nethack одной из самых сложных тестовых сред для алгоритмов машинного обучения с подкреплением. Затем приводится разработанный алготитм для этой задачи и анализируется качество его работы. В заключении суммируются основные результаты полученные в рамках подготовки диссертационного исследования с указанием их новизны и практической значимости. 


\underline{\textbf{Первая глава}} посвящена обзору методов машинного обучения с подкреплением (RL) и применению их в роботике. В ней формулируется задача RL - максимизация метематического ожидания суммарной дисконтированной награды при взаимодействии RL агента и среды. Схематически это взаимодействие изображено на рисунке \ref{fig:rl_setting}.

\begin{figure}[ht]
    \centerfloat{
        \includegraphics[width=0.6\linewidth]{images/rl_setting}
    }
    \caption{Ваимодействик RL агента и среды}\label{fig:rl_setting}
\end{figure}

Математическое ожидание суммарной дисконтированной награды вычисляется по траекториям при условии текущей стратегии агента по формуле: 

\[
E_{\tau \sim \pi(\theta)} [G(\tau)] = E_{\tau \sim \pi(\theta)} [R_0 + \gamma R_{1} + \gamma ^ 2 R_{2} + ...] = E_{\tau \sim \pi(\theta)} [\sum_{t=0}^{T - 1} \gamma ^t R_{t}]
\]

Взаимодействие агента и среды рассматривается в предположении марковского процесса принятия решений. С учетом этого предположения оптимальная стратегия агента зависит только от текущего состояния среды что позволяет использовать уравнение оптимальности Беллмана выражающее связь между оптимальной стратегией в двух последовательных состояниях среды $s$ и $s'$

\[
	V^*(s) = \max_{a \in \mathcal{A}} E(r_{t + 1}(s, a, s') + \gamma V^*(s_{t + 1}))
\]
где $V^*(s)$ (V-функция) - математическое ожидание суммарной дисконтированной награды в состоянии $s$ при условии оптимальной стратегии, а $\gamma$ - коэффициент дисконтирования. Эквивалентно уравнение Беллмана можно записать следующим образом: 

\[
	Q^*(s, a) = E(r_{t + 1}(s, s', a) + \max_{a \in \mathcal{A}} \gamma Q^*(s_{t + 1}, a))
\]

где $Q^*(s, a)$ (Q-функция) - математическое ожидание суммарной дисконтированной награды в состоянии $s$ при условии действия $a$ и следовании оптимальной стратегии в следующем состоянии $s'$.

Далее рассматриваются основные алгоритмы применяемые в машинном обучении с подкреплением. Условно их можно разделить на два больших класса <<off-policy>> и <<on-policy>> методы. Off-policy методы основываются на идее аппроксимации Q-функции c использованием метода временных разностей. Такие методы могут использовать данные полученные стратегией сколь угодно сильно отличающейся от текущей стратегии. Благодаря этому off-policy методы обычно требуют меньше данных, но более подвержены численным не стабильностям при обучении. С другой стороны on-policy методы основываются на идее максимизации суммарной ожидаемой награды благодаря чему они численно более стабильны, но требуют существенно больше данных для обучения. 

Затем рассматриваются подходы применяемые в обучнии с подкреплением в задачах с разреженной наградой. В таких задачах вероятность того, что агент со случайной стратегией найдет не нулевую награду и тем самым получит положительный сигнал для обучения пренебрежимо мала. В этом случае используются подходы называемые внутренней мотивацией (intrinsic motivation). Общая идея данных подходов заключается в том, чтобы давать агенту награды за достижение состояний в которых он раньше не был, что позволит эффективно исследовать среду. 

Последним из подходов используемых в обучении с подкреплением расматривается иерархическое обучение с подкреплением. Данный класс методов направлен на построение иерархии навыков в которой стратегии нижнего уровня решают подзадачи используемые следующими стратегиями для решения более общих задач. 

В конце главы рассматривается применение методов глубокого машинного обучения с подкреплением в задачах управления робототехническими устройствами. По сравнению с классическими методами основанными на обратной кинематике алгоритмы машинного обучения способны самостоятельно адаптироваться к параметрам робота и соответственно работать в условиях когда эти параметры точно не известны или значения получаемые при измерении этих параметров имеют большой разброс. 

\underline{\textbf{Вторая глава}} посвящена разработке метода автоматической настройки оптического интерферометра Маха-Цендера по изображениям с камеры методом машинного обучения с подкреплением. В начале описываются физические принципы работы оптического интерферометра. Затем приводится построенная на основе этих принципов компьютерная модель позволяющаяя моделировать изображения получаемые на камере оптического интерферометра при произвольном расположении оптических элементов - зеркал и линз интерферометра. Примеры изображений полученных с помощью разработанной программы приведены на рисунке \ref{fig:visib_expl}. 

\begin{figure}[ht]
    \centerfloat{
        \includegraphics[width=0.8\linewidth]{images/visib_expl}
    }
    \caption{
    Изображения интерференции для различных положения зеркал полученные с помощью симуляционной программы. (a) Идеально настроенный интерферометр, видность = 1; (b) Слабо расстроенный интерферометр, видность = 0.3; (c) Сильно расстроенный интерферометр, видность = 0.0026. Изображения слева на право соответствуют различным моментам времени.}
\label{fig:visib_expl}
\end{figure}

Далее задача настройки интерферометра сводится к марковскому процессу принятия решений и обучению с подкреплением. Общая схема интерферометра настраеваемого алгоритмом изображена на рисунке \ref{fig:interf_scheme}. На ней RL агент управляет линзой $lens 2$ и светоделителями $bs1$ и $bs2$. 

\begin{figure}[ht]
    \centerfloat{
        \includegraphics[width=0.8\linewidth]{images/interferobot_scheme}
    }
    \caption{Ваимодействик RL агента и среды}\label{fig:interf_scheme}
\end{figure}

Далее рассматриваются подходы к определению пространства действий агента и функции награды. Обосновывается использование в качестве нагады функции $r = vis - log(1 - vis)$, где $vis$ - видность интерференционной картины, основная метрика качества настройки интерферометра. Приводятся результаты обучения агента в симуляции с использованием дисктного и непрерывного пространств действий. Затем рассматриваются рандомизации использованные при обучении агентов в симуляции для успешного переноса на экспериментальную установку. В заключении приводятся результаты тестирования агентов на экспериментальной установке, сравнение качества настройки интерферомета с человеком и анализ стратегии используемой агентами при настройке интерферометра. 

\begin{figure}[ht]
    \centerfloat{
        \includegraphics[width=0.8\linewidth]{images/dqn_vs_td3}
    }
    \caption{Тестирование агентов на экспериментальной установке}\label{fig:interf_test}
\end{figure}

Результаты тестирования приведены на рисунке \ref{fig:interf_test}. По оси абсцисс отложен номер шага настройки, по оси ординат отложена достигнутая видность. Результаты усреднены по 100 эпизодам. Из рисунка \ref{fig:interf_test} видно, что оба метода достигают хорошего значения видности интерференционной картины, но агент TD3 использующий непрерывное пространство действий работает быстрее и достигает лучшей видности чем DQN агент использующий дискретное пространство действий. 


Разультаты работы опубликованы в статьях \cite{confbib1, confbib2} на ведущих научных конференциях <<Neural information processing systems (NeurIPS)>> и <<Conference on Robot Learning (CoRL)>> также на программу автоматической настройки интерферометра по изображениям с камеры с импользованием машинного обучения получен РИД \cite{progbib1}.

\underline{\textbf{Третья глава}} посвящена исследованию методов машинного обучения с подкреплением для использования в среде Nethack. Среда Nethack основанна на одноименной игре и предложена в качестве теста для алгоритмов машинного обучения в 2020 году \cite{nethack}. В этой игре агент путешествует по процедурно генерируемому подземелью. Игра имеет ascii интерфейс изображенный на рисунке \ref{fig:nethack}.

\begin{figure}[ht]
    \centerfloat{
        \includegraphics[width=0.8\linewidth]{images/nethack}
    }
    \caption{Игра Nethack}\label{fig:nethack}
\end{figure}

Основную сложность для алгоритмов обучения с подкреплением составляют сочетание различных типов данных в состоянии (изображения, текстовые описания, табличные данные), редкая функция награды и комбинаторно большое пространство действий. Далее приводится разработанный метод сочетающий в себе машинное обучение с подкреплением, алгоритмы навигации на графах и алгоритмы основанные на экспертных знаниях. 

\SetKwComment{Comment}{/* }{ */}
\SetKw{Continue}{continue}

\begin{algorithm}
\caption{RAPH algorithm}\label{alg:one}
\KwData{view\_distance, agent}
$state \gets env.reset()$\;
$done \gets False$\;

\While{not done}{
  action\_queue = senses.update(state)\;

  \If{action\_queue} {
   state, reward, done, info = env.step(action\_queue) \Comment*[r]{We have a prompt to response}
   \Continue
  }

  level.update(state)\;
  inventory.update(state)\;
  hero.update(state)\;
  \eIf{monster\_distance \textless view\_distance}{
    action\_queue = agent.act(preprocessed\_state)\;
  }{
    action\_queue = first\_fit(non-rl-actions)\Comment*[r]{Select non-rl action on first-fit basis}
  }
  state, reward, done, info = env.step(action\_queue)\;
}

\label{alg:raph}
\end{algorithm}

Данный подход позволил превзойти остальные подходы с использованием обучения с подкреплением в этой задаче. Результаты были опубликованы на одной из ведущих конференции по машинному обучению NeurIPS в рамках трека посвященного соревнованиям \autocite{confbib3}.


\FloatBarrier
\pdfbookmark{Заключение}{conclusion}                                  % Закладка pdf
В \underline{\textbf{заключении}} приведены основные результаты работы, которые заключаются в следующем:
%% Согласно ГОСТ Р 7.0.11-2011:
%% 5.3.3 В заключении диссертации излагают итоги выполненного исследования, рекомендации, перспективы дальнейшей разработки темы.
%% 9.2.3 В заключении автореферата диссертации излагают итоги данного исследования, рекомендации и перспективы дальнейшей разработки темы.
\begin{enumerate}
  \item Была разработана компьютерная модель оптического интерферометра Маха-Цендера. На основании компьютерной модели была разработана среда для обучения агентов машинного обучения с подкреплением настройке оптического интерферометра. Среда позволяет моделировать интерференционные картины, получаемые в интерферометре Маха-Цендера с произвольным расположением зеркал и линз. 
  \item Процесс настройки интерферометра был представлен в виде Марковского процесса принятия решений, была определена функция награды, пространства состояний и действий агентов. Предложенная функция награды позволяет различать состояния с высокой видностью, благодаря чему агент способен производить точную настройку интерферометра. Было предложено преобразование пространства действий, позволяющее агенту, оперирующему непрерывными действиями, одинаково хорошо выбирать действия с разными порядками амплитуды. 
  \item Были разработаны и реализованы алгоритмы машинного обучения с подкреплением, способные оперировать действиями различного масштаба и устойчивые к оптическим шумам. 
  \item Разработанные алгоритмы были обучены и успешно протестированы на экспериментальной установке. Качество настройки интерферометра у агента, оперирующего непрерывным пространством действий, оказалось существенно выше, чем у человека. 
  \item Был разработан метод обучения стратегии для управления движением шагающего робота с заданной линейной и угловой скоростью. Благодаря предложенной функции награды получилось обучить стратегию, способную успешно управлять движением робота Unitree A1 с заданной скоростью. 
  \item Был разработан гибридный нейро-символьный метод для управления виртуальным агентом в среде NetHack, включающий в себя обучение с подкреплением. Предложенный метод показал свою эффективность в рамках соревнования Neurips NetHack Challange и занял первое место среди алгоритмов, использующих нейронные сети.
\end{enumerate}

\newpage


\pdfbookmark{Литература}{bibliography}                                % Закладка pdf

\ifdefmacro{\microtypesetup}{\microtypesetup{protrusion=false}}{} % не рекомендуется применять пакет микротипографики к автоматически генерируемому списку литературы
\urlstyle{rm}                               % ссылки URL обычным шрифтом
\ifnumequal{\value{bibliosel}}{0}{% Встроенная реализация с загрузкой файла через движок bibtex8
    \renewcommand{\bibname}{\large \bibtitleauthor}
    \nocite{*}
    \insertbiblioauthor           % Подключаем Bib-базы
    %\insertbiblioexternal   % !!! bibtex не умеет работать с несколькими библиографиями !!!
}{% Реализация пакетом biblatex через движок biber
    % Цитирования.
    %  * Порядок перечисления определяет порядок в библиографии (только внутри подраздела, если `\insertbiblioauthorgrouped`).
    %  * Если не соблюдать порядок "как для \printbibliography", нумерация в `\insertbiblioauthor` будет кривой.
    %  * Если цитировать каждый источник отдельной командой --- найти некоторые ошибки будет проще.
    %
    %% authorvak
    \nocite{vakbib1}%
    \nocite{vakbib2}%
    %
    %% authorwos
    \nocite{wosbib1}%
    %
    %% authorscopus
    \nocite{scbib1}%
    %
    %% authorpathent
    \nocite{patbib1}%
    %
    %% authorprogram
    \nocite{progbib1}%
    %
    %% authorconf
    \nocite{confbib1}%
    \nocite{confbib2}%
    \nocite{confbib3}%
    %
    %% authorother
    \nocite{bib1}%
    \nocite{bib2}%

    \ifnumgreater{\value{usefootcite}}{0}{
        \begin{refcontext}[labelprefix={}]
            \ifnum \value{bibgrouped}>0
                \insertbiblioauthorgrouped    % Вывод всех работ автора, сгруппированных по источникам
            \else
                \insertbiblioauthor      % Вывод всех работ автора
            \fi
        \end{refcontext}
    }{
        \ifnum \totvalue{citeexternal}>0
            \begin{refcontext}[labelprefix=A]
                \ifnum \value{bibgrouped}>0
                    \insertbiblioauthorgrouped    % Вывод всех работ автора, сгруппированных по источникам
                \else
                    \insertbiblioauthor      % Вывод всех работ автора
                \fi
            \end{refcontext}
        \else
            \ifnum \value{bibgrouped}>0
                \insertbiblioauthorgrouped    % Вывод всех работ автора, сгруппированных по источникам
            \else
                \insertbiblioauthor      % Вывод всех работ автора
            \fi
        \fi
        %  \insertbiblioauthorimportant  % Вывод наиболее значимых работ автора (определяется в файле characteristic во второй section)
        \begin{refcontext}[labelprefix={}]
            \insertbiblioexternal            % Вывод списка литературы, на которую ссылались в тексте автореферата
        \end{refcontext}
        % Невидимый библиографический список для подсчёта количества внешних публикаций
        % Используется, чтобы убрать приставку "А" у работ автора, если в автореферате нет
        % цитирований внешних источников.
        \printbibliography[heading=nobibheading, section=0, env=countexternal, keyword=biblioexternal, resetnumbers=true]%
    }
}
\ifdefmacro{\microtypesetup}{\microtypesetup{protrusion=true}}{}
\urlstyle{tt}                               % возвращаем установки шрифта ссылок URL
