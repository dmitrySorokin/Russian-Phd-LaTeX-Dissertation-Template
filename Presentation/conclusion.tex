\begin{frame}
\frametitle{Положения, выносимые на защиту}
\begin{enumerate}
  \item Разработаны алгоритмы, подбора параметров симуляции и масштабирования пространства действий. 
  \item На основе предложенных алгоритмов реализованы методы настройки оптического интерферометра основанные на обучении с подкреплением.
  \item Разработан программно-аппаратный комплекс Интерферобот. Скорость и точность настройки с использованием разработанного комплекса существенно превосходят ручную настройку.
  \item Предложен метод обучения по расписанию для многозадачного агента. Метод использован для обучения робота Unitree A1 решать различные задачи перемещения. 
  \item Предложен метод объединения обучаемого и алгоритмического подходов в рамках одного агента. С использованием предложенного метода реализован гибридный агент для среды NetHack.
    \end{enumerate}
\end{frame}
\note{
    Проговаривается вслух научная новизна
}

\begin{frame}
    \frametitle{Свидетельство о регистрации программы}
    \begin{figure}[h]
        \centering
        \includegraphics[height=0.7\textheight]{interferobot_rid.pdf}
    \end{figure}
\end{frame}
\note{
    Получено свидетельство о регистрации разработанной программы \textsc{Hello~world™}.
}

%\begin{frame}
%    \frametitle{Акт о внедрении}
%    \begin{figure}[h]
%        \centering
%        \fbox{
%            \begin{minipage}[t]{0.4\linewidth}
%                %\includegraphics[width=\linewidth]{implementation}
%            \end{minipage}
%        }
%    \end{figure}
%\end{frame}
%\note{
%    Получен акт о внедрении.
%}

%\begin{frame} % публикации на одной странице
\begin{frame}[t,allowframebreaks] % публикации на нескольких страницах
\frametitle{Публикации результатов диссертационной работы}

\begin{enumerate}
\fontsize{10pt}{10pt}\selectfont
    \item Interferobot: aligning an optical interferometer by a reinforcement learning agent. \textcolor{gray}{/. –– }\textbf{D. Sorokin}\textcolor{gray}{, A. Ulanov, E. Sazhina, A. Lvovsky // Advances in Neural Information Processing Systems. Vol. 33 / ed. by H. Larochelle, M. Ranzato, R. Hadsell, M. F. Balcan, H. Lin. –– Curran Associates, Inc., 2020. –– P. 13238––13248.} (CORE A*)
    \item Aligning an optical interferometer with beam divergence control and continuous action space. \textcolor{gray}{/. –– S. Makarenko,} \textbf{D. I. Sorokin}\textcolor{gray}{, A. Ulanov, A. Lvovsky // Proceedings of the 5th Conference on Robot Learning. Vol. 164 / ed. by A. Faust, D. Hsu, G. Neumann. –– PMLR, 2022. –– P. 918––927. –– (Proceedings of Machine Learning Research)}
    \item Insights From the NeurIPS 2021 NetHack Challenge. \textcolor{gray}{/. –– E. Hambro, S. Mohanty, D. Babaev, M. Byeon, D. Chakraborty, E. Grefenstette, M. Jiang, J. Daejin, A. Kanervisto, J. Kim, S. Kim, R. Kirk, V. Kurin, H. Küttler, T. Kwon, D. Lee, V. Mella, N. Nardelli, I. Nazarov, N. Ovsov, J. Holder, R. Raileanu, K. Ramanauskas, T. Rocktäschel, D. Rothermel, M. Samvelyan,} \textbf{D. Sorokin}\textcolor{gray}{, M. Sypetkowski, M. Sypetkowski // Proceedings of the NeurIPS 2021 Competitions and Demonstrations Track. Vol. 176 / ed. by D. Kiela, M. Ciccone, B. Caputo. –– PMLR, 2022. –– P. 41––52. –– (Proceedings of Machine Learning Research)} (CORE A*)
    \item Learning Various Locomotion Skills from Scratch with Deep Reinforcement Learning. \textcolor{gray}{/. –– }\textbf{D. I. Sorokin}\textcolor{gray}{, D. L. Babaev // Advances in Neural Computation, Machine Learning, and Cognitive Research VI / ed. by B. Kryzhanovsky, W. Dunin-Barkowski, V. Redko, Y. Tiumentsev. –– Cham : Springer International Publishing, 2023. –– P. 322––329.} (SCOPUS)
\end{enumerate}
    
\end{frame}
\note{
    Результаты работы опубликованы в N печатных изданиях,
    в~т.\:ч. M реферируемых изданиях.
}

\begin{frame}[t,allowframebreaks]
\frametitle{Презентации на международных конференциях}
\vspace{20pt}
\setlength{\leftmargini}{0cm}


\begin{columns}
\column{0.6\linewidth}

\begin{itemize}
\setlength\itemsep{1em}
    \item[] {\color{orange}Neural Information Processing Systems}\\
    NeurIPS 2020, (CORE A*)\\
    доклад был отмечен как spotlight
    \item[] {\color{orange}Conference on Robot Learning}\\
    CoRL 2021
    \item[] {\color{orange}Neural Information Processing Systems}\\
    NeurIPS 2021, (CORE A*)\\
    Competition track
    \item[] Международная научно-техническая конференция Нейроинформатика 2022 (SCOPUS)
\end{itemize} 
\column{0.4\linewidth}
\includegraphics[width=1\linewidth]{Presentation/images/logo/neurips.png}
\includegraphics[width=1\linewidth]{Presentation/images/logo/corl.png}
\includegraphics[width=1\linewidth]{Presentation/images/logo/neurips.png}
\includegraphics[width=1\linewidth]{Presentation/images/logo/neuroinfo.png}
\end{columns}

\begin{columns}
\column{0.7\linewidth}
\begin{itemize}
\vspace{40pt}
\setlength\itemsep{1.5em}
    \item[] Annual International Laser Physics Workshop\\
    LPHYS 2021
    \item[] International Conference on Quantum Technologies\\
    ICQT 2021
    \item[] Machine Learning Summer School\\
    MLSS 2019
\end{itemize} 

\column{0.3\linewidth}
\vspace{50pt}
\begin{itemize}
\setlength\itemsep{1.5em}
    \item[] {\includegraphics[width=1\linewidth]{Presentation/images/logo/lphys.png}}
    \item[] {\includegraphics[width=1\linewidth]{Presentation/images/logo/icqt.png}}
    \item[] {\includegraphics[width=1\linewidth]{Presentation/images/logo/mlss.png}}
\end{itemize}

\end{columns}

\end{frame}
\note{
    Работа была представлена на ряде конференций.
}

\begin{frame}[plain, noframenumbering] % последний слайд без оформления
    \begin{center}
        \Huge
        Спасибо за внимание!
    \end{center}
\end{frame}
