%% Согласно ГОСТ Р 7.0.11-2011:
%% 5.3.3 В заключении диссертации излагают итоги выполненного исследования, рекомендации, перспективы дальнейшей разработки темы.
%% 9.2.3 В заключении автореферата диссертации излагают итоги данного исследования, рекомендации и перспективы дальнейшей разработки темы.
\begin{enumerate}
  \item Была разработана компьютерная модель оптического интерферометра Маха-Цендера. На основании компьютерной модели была разработана среда для обучения агентов машинного обучения с подкреплением настройке оптического интерферометра. Среда позволяет моделировать интерференционные картины получаемые в интерферометре Маха-Цендера с произвольным расположением зеркал и линз. 
  \item Процесс настройки интерферометра был представлен в виде Марковского процесса принятия решений, была определена функция награды, пространства состояний и действий агентов. Предложенная функция награды позволяет различать состояния с высокой видностью благодаря чему агент способен производить точную настройку интерферометра. Было предложено преобразование пространства действий позволяющее агенту оперирующему непрерывными действиями одинаково хорошо выбирать действия с разными порядками амплитуды. 
  \item Алгоритмы машинного обучения с подкреплением были реализованы и обучены настройке интерферометра в симуляции. Благодаря использованию в процессе обучения предложенного набора шумов, обученные агенты были успешно протестированы на экспериментальной установке. Качество настройки интерферометра у агента оперирующего непрерывным пространством действий оказалось существенно выше, чем у человека. 
  \item Был разработан гибридный нейро-символьный метод для управления виртуальным агентом в среде NetHack включающий в себя обучение с подкреплением. Предложенный метод показал свою эффективность в рамках соревнования Neurips NetHack Challange и занял первое место среди алгоритмов использующих нейронные сети.
  \item TODO Unitree A1
\end{enumerate}

\newpage
