%% Согласно ГОСТ Р 7.0.11-2011:
%% 5.3.3 В заключении диссертации излагают итоги выполненного исследования, рекомендации, перспективы дальнейшей разработки темы.
%% 9.2.3 В заключении автореферата диссертации излагают итоги данного исследования, рекомендации и перспективы дальнейшей разработки темы.
\begin{enumerate}
  \item Была разработанна компьютерная модель оптического интерферометра Маха-Цендера. Процесс настройки интерферометра был представлен в виде марковского процесса принятия решений на основании которого была разработана среда для обучения агентов машинного обучения с подкреплением по настройке оптического интерферометра. 
  \item На основе среды были разработаны методы основанные на обучении с подкреплением, которые позволили выучить алгоритм настройки интерферометра по изображениям с камеры. 
  \item Разработанные методы были успешно протестированы при настройке экспериментальной установки интерферометра. 
  \item Был разработан метод включающий в себя обучения с подкреплением для управлением виртуальным агентом в среде Nethack. 
\end{enumerate}

\newpage
