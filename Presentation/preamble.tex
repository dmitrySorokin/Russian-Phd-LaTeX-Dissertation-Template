\begin{frame}[noframenumbering,plain]
    \setcounter{framenumber}{1}
    \maketitle
\end{frame}

\begin{frame}
    \frametitle{Область применения методов RL ограничена:}
    \begin{itemize}
        \item Большой объем данных для обучения
        \item Sim2Real gap при переносе агента обученного в симуляции на физическую установку
       \item Сложности с объединением RL и классических алгоритмов в рамках иерархического агента 
        \item Сходимость стратегии к локальному оптимуму при не оптимальной функции награды
    \end{itemize}
\end{frame}

\begin{frame}
    \frametitle{Положения, выносимые на защиту}
    \begin{itemize}
        \item Метод обучения с подкреплением способный
оперировать действиями различного масштаба и устойчивый к
оптическим шумам
        \item Программно-аппаратный комплекс автоматической настройки оптического интерферометра
        \item Иерархический алгоритм комбинирующий алгоритмический и нейросетевой подходы и его применение для управления агентом в среде NetHack
        \item Метод обучения стратегии для управления движением шагающего робота с заданной линейной и угловой скоростью
    \end{itemize}
\end{frame}
\note{
    Проговариваются вслух положения, выносимые на защиту
}

\begin{frame}
    \frametitle{Содержание}
    \tableofcontents
\end{frame}
\note{
    Работа состоит из четырёх глав.

    \medskip
    В первой главе \dots

    Во второй главе \dots

    Третья глава посвящена \dots

    В четвёртой главе \dots
}
